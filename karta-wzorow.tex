\documentclass[10pt,a4paper,twocolumn]{article}

\usepackage{multicol}
\usepackage{amsbsy, amssymb, latexsym, amsmath, braket}
\usepackage[tiny]{titlesec}
\usepackage[hmargin=0.5cm,vmargin=0.7cm]{geometry}
\usepackage[utf8x]{inputenc}
\usepackage{polski}
\usepackage{scalefnt}
\usepackage[yyyymmdd,hhmmss]{datetime}
\usepackage{commath}

% Potrzebne do algorytmu Euklidesa.
\usepackage{tikz}
\usetikzlibrary{tikzmark}

\newcommand{\angles}[1]{\left\langle #1 \right\rangle}


\newcommand{\entry}{$\bullet$\hspace{0.15em}}
\newcommand{\subentry}{$\circledcirc$\hspace{0.15em}}
% https://tex.stackexchange.com/a/7045/80219
\newcommand{\textsubentry}[1]{\tikz[baseline=(char.base)]{
            \node[shape=circle,draw,inner sep=1pt] (char) {#1};\hspace{0.15em}}}

% Automatyczne generowanie listy liczby pierwszych:
% https://tex.stackexchange.com/a/134320/80219
\makeatletter
\def\primes#1#2{{%
  \def\comma{\def\comma{, }}%
  \count@\@ne\@tempcntb#2\relax\@curtab#1\relax
  \@primes}}
\def\@primes{\loop\advance\count@\@ne
\expandafter\ifx\csname p-\the\count@\endcsname\relax
\ifnum\@tempcntb<\count@\else
  \ifnum\count@<\@curtab\else\comma\the\count@\fi\fi\else\repeat
\@tempcnta\count@\loop\advance\@tempcnta\count@
\expandafter\let\csname p-\the\@tempcnta\endcsname\@ne
\ifnum\@tempcnta<\@tempcntb\repeat
\ifnum\@tempcntb>\count@\expandafter\@primes\fi}
\makeatother

\titlespacing{\section}{0pt}{0pt}{0pt}
\titlespacing{\subsection}{0pt}{0pt}{0pt}
\titlespacing{\subsubsection}{0pt}{0pt}{0pt}

% Wyłącz numerowanie stron.
\pagenumbering{gobble}

\setlength{\parindent}{0pt}
% Odległość pomiędzy liniami. Zmniejsz, jeżeli brakuje miejsca.
\setlength{\parskip}{0.5ex}

\title{Karta wzorów z matematyki dyskretnej}

\begin{document}
% Rozmiar czcionki.
\scalefont{.8}

\text{\tiny{
    Wersja z \today\ o \currenttime\ (\pdfmdfivesum file{./karta-wzorow.tex})
}}

\section{Sumy}

\entry
${S_n = \sum^{n}_{k = 0} a \cdot x^k} = {a \frac{1 - x^{n}}{1 - x}}$;

\entry
$\sum_{i=0}^{n-1}\frac{1}{(i+1)(i+2)} = \frac{n}{n+1}$;

\entry
$\sum_{i=0}^{n}(-1)^i i^2 = (-1)^n\frac{n(n+1)}{2}$;

\subsection{Sumowanie przez części}

\entry
${\Delta f(x) = f(x + 1) - f(x)}$;
\entry
${\Delta x^{\underline{n}}} = {n x^{\underline{n - 1}}}$;
\entry
${\mathcal{S}^{b}_{a} f}={g |^b_a} = {\sum^{b - 1}_{k = a} f(k)}$;
\entry
${\Delta (rt)} = {r \Delta t + E t \Delta r \text{, gdzie } E t(x)} =
  {t(x + 1)}$;
  \entry
${\mathcal{S}^{b}_{a} r \Delta t}= {r t |^b_a - \mathcal{S}^b_a E t \Delta r}$;

\subsection{Dwumian}

\entry
${\binom{n}{k} = \binom{n}{n - k} = {\frac{n^{\underline{k}}}{k!}}}$;
\entry
${\sum_{i = 0}^n \binom{n}{i} = 2^n}$;
\entry
${\sum_{i = 0}^{n} (-1)^i \binom{n}{i} = [ n = 0 ]}$;

\entry
$\binom{n}{k} = \frac{n}{k} \binom{n - 1}{k - 1}$;

\entry
${(x + y)^n = \sum_{i \geq 0} \binom{n}{i} x^i y^{n-i}, n \in \mathbb{N}}$;

\entry
${(1 + x)^r = \sum_{i \geq 0} \binom{r}{i} x^i, r \in \mathbb{R}, |x| < 1}$;

\entry
$(x + y + z)^n =
  \sum_{0 \leq a, b \leq n, a+b \leq n} \binom{n}{a,b} x^a y^b z^{n-a-b}$;

\entry
$\binom{n}{a,b} = \binom{n}{a}\binom{n-a}{b} = \frac{n!}{a! b! (n - a - b)!}$;

\entry
$\binom{n}{k}\binom{k}{i} = \binom{n}{i}\binom{n-i}{k-i}$;

\entry
$\binom{n}{k} = \begin{cases}
    1, k = 0 \text{ lub } k = n \\
    \binom{n - 1}{k} + \binom{n - 1}{k - 1}, 0 < k < n
\end{cases}
$;

% Tożsamość Cauchy'ego.
\entry
Toż. (Cauchy): $\sum^k_{i=0}\binom{n}{i}\binom{m}{k - i} = \binom{n + m}{k}$;

\entry
$\text{$\sum$ równoległe: } \sum^k_{i=0} \binom{y + i}{i} =
  \binom{y + k + 1}{k}$;

\entry
$(-1)^i\binom{x}{i} = \binom{i - 1 - x}{i} \text{, bo }  x^{\underline{i}} =
  (-1)^i(i - 1 - x)^{\underline{i}}$;

\entry
$a_n = \sum_i\binom{n}{i}(-1)^i b_i \iff b_n = \dots$;

% Inwersje.
\subsection{Inwersje}

\entry
$\Pi = 1^{\lambda_1}2^{\lambda_2}\dots n^{\lambda_n}$, to $sgn(\Pi) =
  (-1)^{\Sigma_i\lambda_{2i}}$;

% Liczby Stirlinga I rodzaju.
\subsection{Liczby Stirlinga}

\entry
${n \brack k}$ - podziały $n$-zbioru na $k$ cykli (kolejność w cyklu istotna z
  dokładnością do cyklicznego przesunięcia);

\entry
${n \brack 0} = [ n = 0 ]$;
\entry
${n \brack 1} = (n - 1)!, n > 0$;
\entry
$\sum_k{n \brack k} = n!$;

\entry
${n \brack k} = (n-1){n-1 \brack k}+{n-1 \brack k-1}$ dla ${k > 0}$;

% Liczby Stirlinga II rodzaju.
\entry
${n \brace k}$ - podziały $n$-zbioru na $k$ bloków;

\entry
${n \brace 0} = [n = 0]$;
\entry
${n \brace 1} = 1$;
\entry
${n \brace 2} = 2^{n-1}-1$;

\entry
${n \brace k} = k{n-1 \brace k} + {n-1 \brace k-1}$ dla ${k > 0}$;

% Własności liczb Stirlinga.
\entry
$x^n = \sum_k{n \brace k} x^{\underline{k}}$;

\entry
$x^{\overline{n}} = \sum_k{n \brack k}x^k$;

\entry
$(-x)^{\overline{k}} = (-1)^kx^{\overline{k}}$;

\entry
$x^n = \sum_k {n \brace k} (-1)^{n-k} x^{\overline{k}}$;

\entry
$x^{\underline{n}} = \sum_k {n \brack k} (-1)^{n-k} x^{k}$;

\entry
$x^n = \sum_{i,k} {n \brace i}{i \brack k}(-1)^{n-i}x^k$;

\entry
$\sum_i{n \brace i}{i \brack k}(-1)^{n-i} = [n=k] =
  \sum_i{n \brack i}{i \brace k}(-1)^{n-i}$;

\entry
$a_n = \sum_i{n \brace i}(-1)^ib_i \iff b_n = \sum_i{n \brack i}(-1)^ia_i$;

\input{src/funkcje-tworzace.tex}
\section{Magiczne liczby}

\begin{tiny}
  \primes{1}{2719}
\end{tiny}

\subsection{Liczby Catalana}

\entry
$C_n$ - liczba nawiasowań w wyr. $x_0 \cdot x_1 \cdot \dots \cdot x_n$;

\entry
$C_n = \sum_kC_kC_{n-1-k} + [n=0]$,
  skąd $C(x) = \sum_n C_nx^n = x(C(x))^2 + 1, C(0) = 1$,
  więc $C(x) = \frac{1 - \sqrt{1-4x}}{2x}$, a zatem $C(x) =
  \sum_{k\geq 0} \frac{1}{k+1}\binom{2k}{k}x^k$;

\subsection{Liczby Bella}

\entry
${B_n = \sum_k{n \brace k}}$ - łączna l. podziałów $n$-zbioru na bloki;

\entry
${B_{n+1} = \sum_k \binom{n}{k} B_k}$, skąd $B'(x) = e^xB(x), B(0) = 1$,
  gdzie $B(x) = \sum_n B_nx^n/n!$. Całkujemy i dostajemy $ln B = e^x + c$,
  skąd $B(x) = e^{e^x-1} =
  \sum_{n=0}^\infty \left ( {1 \over e} \sum_{k=0}^\infty {k^n \over k!}\right)
  {x^n \over n!}$;

% Enumeratory.
\subsection{Enumeratory kombinacji}

\entry
$(1+t)\cdot\overset{n}{\ldots}\cdot(1+t) = \sum_r\binom{n}{r}t^r$;

\entry
Kombinacja z dowolnymi powt.: $(1+t+t^2+\dots)^n=(1-t)^{-n}=
  \sum_r\binom{-n}{r}{-t}^r=\sum_r\binom{n+r-1}{r}t^r$;

\entry
Każdy elem. co najmniej raz: $(t+t^2+\dots)^n = t^n\sum_r {n+r-1 \over r}t^r =
  \sum^\infty_{r=n}\binom{r-1}{n-1}t^r$;

\subsection{Enumeratory permutacji}

\entry
$r$-perm. bez powt.: $(1+t)^n = \sum_rn^{\underline{r}}{t^r \over r!}$;

\entry
Dow. l. powt.: ${(1+t+t^2/2! + \dots)^n} = {(e^{t})^n} =
  {\sum_rn^r{t^r \over r!}}$;

\entry
Każdy elem. co najmniej raz: $(t + t^2/2! + \dots)^n = (e^t - 1)^n =
  \sum^\infty_{r=0}\frac{t^r}{r!}\sum^n_{j=0}\binom{n}{j}(-1)^j(n-j)^r$;

\entry
Ciąg $A$, $B$, $C$ długości $r$ t., że $\#A>0$, $2\mid\#B$. Traktuj jako r-permu.
  z powt.:
  $\overbrace{(e^t - 1)}^A\cdot\overbrace{((e^t - e^{-t})/2)}^B
  \cdot\overbrace{e^t}^C =
  \sum_{r \geq 1} {1 \over 2}(3^r - 2^r - 1){t^r \over r!}$;

  \entry 
Liczba ścieżek w prostokącie $n \times m$: $\binom{n+m}{n}$

\section{Zasada włączania-wyłączania}

\entry
Liczba elem. o $j$ własn.: $S_j =
  \sum_{1 \leq i_1 < \dots < i_j \leq n} |A_{i_1} \cap \dots \cap A_{i_j}|$;

\entry
Liczba elem. o dokładnie $k$ własn.:
  $D(k) = \sum_{j\geq k} \binom{j}{k} (-1)^{j-k}S_j$;

\entry 
W szczególności, liczba elem. bez własności: $D(0) = \sum_{j\geq 0}(-1)^jS_j$

\entry
Zliczanie $n$-nieporządków: $A_i = \left\{n\text{-perm.} | f(i) = i \right\}$.
  Wtedy $|A_{i_1} \cap \dots \cap A_{i_j}| = (n-j)!$, zatem
  $D(0) = \sum^n_j(-1)^j\binom{n}{j}(n-j)!=n!\sum^n_j(-1)^j{1 \over j!}$;

\section{Wieżomiany}

\entry
$Z_n \stackrel{\text{ozn}}{=} {1,...,n}\text{, plansza: }B \subseteq Z_n\times Z_m$;

\entry
$\text{skojarzenie: } M \subseteq Z_n \times Z_m \text{t.że }\left\langle p,q \right\rangle , \left\langle r,s\right\rangle \in M \Rightarrow p \neq r \& q \neq s$;

\entry
$S_k(B) \text{- zbiór k-skojarzeń zawartych w B, }r_k(B) = |S_k(B)|$;

\entry
$R_B(x) = \sum_{k=0}^{\infty}r_k(B)x^k \text{- wieżomian dla B}$;

\entry
Permutacje wierszy i kolumn nie zmieniają wieżomianu;

\entry
$B = B_1 \oplus B_2 \text{ jeśli } B=B_1 \cup B_2 \text{ oraz } \forall \left\langle p,q\right\rangle \in B_1,\left\langle r,s\right\rangle \in B_2 \text{,} p\neq q \wedge q\neq s $;

\entry
$B = B_1\oplus B_2 \Rightarrow R_B(x) = R_{B_1}(x) \cdot R_{B_2}$;

\entry
Dla $\alpha = \left\langle p, q\right\rangle \in B$: $ B_\alpha^- = B \backslash \{\alpha\}\text{, } B_\alpha^* = B$ bez p-tego wiersza i q-tej kolumny;

\entry
$\alpha \in B \Rightarrow R_B(x) = R_{B_\alpha^-}(x) + x \cdot R_{B_\alpha^*(x)}$;

\entry
Jeśli $C$ jest dopełnieniem planszy $B \subseteq Z_n \times Z_m$, to \\ $r_k(B)=\frac{1}{(m-k)!}\sum_{i=0}^k(-1)^i\binom{n-i}{k-i}(m-i)!r_i(C)$;

\entry
W szczególności, dla $k=n=m$ mamy \\
$r_n(B) = \sum_{i=0}^n(-1)^i(n-i)!r_i(C)$;
\input{src/podzialy-liczby.tex}
\section{Grafy}

\entry
Lem. o uściskach dł.: $\sum_{v\in V} deg(v) = 2|E|$;

\entry
$H$ podgrafem indukowanym $G
  \iff \forall_{u,v \in V[H]} \set{u, v} \in E[G] \implies \set{u, v}\in E[H]$;

\entry
Droga nie powtarza krawędzi, a ścieżka wierzchołków;

\entry
$G$ spójny $\iff \forall_{u,v\in V[G]} \exists e_{uv}$;

\entry
$G$ $k$-reguralny $\iff \forall_v deg(v) = k$;

\entry
$G$ dwudzielny, gdy $V[G] = V_1 \cup V_2, V_1 \cap V_2 = \emptyset$ i każda
  krawędź ma jeden koniec w $V_1$, a drugi w $V_2$;

\entry
$K_{|V|, |U|}$ pełny dwudzielny, gdy $E=\set{\set{v, u}, v \in V, u \in U}$;

\entry
$v$ rozcinający, gdy usunięcie $v$ zwiększa l. spójnych s.;

\entry
Tw.: $G$ dwudzielny $\iff$ nie zawiera cykli nieparz. dł.;

\subsection{Cykle}

\entry
C. E. --- krawędzie; C. H. --- wierzchołki; Graf h. --- graf z c. H.;

\entry
Tw. Eulera: $G$ ma c. E. $\iff \forall_{v \in V[G]} 2|deg(v)$;

\entry
Tw.: Silnie spójny $G$ ma skierowany c. E.
  $\iff \forall_{v \in V[G]} deg_{in}(v) = deg_{out}(v)$;

\entry
$G$ ma c. H., to po usunięciu dow. k wierz. rozpada się na co najw.
  k spójnych s.;

\entry
$G = \angles{V, U; E}$ dwudzielny ma c. H., to $|V|=|U|$;

\entry
Tw.: Każdy turniej jest półhamilton. (zawiera ś. H.);

\entry
Tw.: Turniej ma c. H. $\iff$ jest silnie spójny;

\entry
Tw.: Turniej spójny, to ma c. H.;

\entry
Tw. (Ore): $n=|V|\geq 3$ i
  $\forall_{\set{v, w} \not\in E} deg(v) + deg(w) \geq n$, to $G$ ma c. H.;

\entry
Każdy turniej hamiltonowski jest silnie spójny;

\entry
$G$ silnie spójny $\Rightarrow$ $G$ zawiera skierowany $k\text{-cykl}$;

\subsection{Drzewa}

\entry
Tw. (Cayley): Jest $n^{n-2}$ etykietowanych drzew $n$-wierzchołkowych;

\entry
Równoważne są:\\
G jest drzewem,\\
każde dwa wierzchołki w G są połączone dokładnie jedną droga,\\
G jest minimalny spójny,\\
G jest maksymalny acykliczny,\\
G jest spójny i $|V| = |E| + 1$ 

\subsection{Planarność}

\entry
Wz. Eulera: $n - m + f = 2$, gdzie $m = |E[G]|$;

\entry
Tw.: W grafie plan. z $n \geq 3$ mamy $m \leq 3n -6$;

\entry
Mocjiejszym twierdzeniem jest gdy graf nie zawiera trójkątów $m \leq 2n - 3$;

\entry
Tw. Kuratowskiego: $G$ nieplan. $\iff G$ zawiera podgraf homeomorficzny
  z $K_{3,3}$ lub z $K_5$ (homeomorficzny, czyli
  izomorficzny po ew. dołożeniu wierzchołków na krawędziach);

\entry
$G$ planarny $\implies \exists_{v \in G[V]} deg(v) \leq 5$;

\subsection{Kolorowanie wierzchołków}

\entry
Kolorowanie $G$ za pomocą $k$ kolorów to
  $f: V[G] \rightarrow \set{1, \dots, k}$ t., że $f(u) \neq f(v)$ dla
  $\set{u, v} \in E[G]$. Najm. k t., że $\exists k$-kolorowanie $G$ to liczba
  chromatyczna $\chi (G)$.

\entry
$\chi(G) \leq 2 \Leftrightarrow G$ dwudzielny;

\entry
$\chi(G) \leq k \Leftrightarrow \chi(B) \leq k$ dla każdej dwuspójnej s. $B$
  grafu $G$;

\entry
Tw. o 4 barwach: $G$ plan. $\implies\chi(G)\leq 4$;

\entry
Tw. Brooksa: $G$ spójny, nie cykl nieparz. dł., nie klika, to
  $\chi(G) \leq \Delta$, gdzie $\Delta$ to maks. stop. wierz. w $G$;

\entry
Tw.: $\chi(G) \leq \Delta + 1$;

\entry
$f_G(t)$ --- wielomian chrom. (liczba kolorowań $G$ za pomocą $t$ kolorów);

\entry
W. ch.:
\entry
$K_n$ --- $t^{\underline{n}}$,
\entry
$\overline{K_n}$ --- $t^n$,
\entry
$\text{Drzewo}_n$ --- $t(t-1)^{n-1}$,
\entry
$\text{Cykl}_n$ --- $(t-1)^n + (-1)^n(t-1)$,
\entry
$K_{n,m}$ --- $\sum_{a,b}{n \brace a}{m \brace b}t^{\underline{a+b}}$;

\entry
Tw.: $e = \set{v, w} \not\in E[G]$, to $f_G(t)=f_{G\cup e}(t) + f_{G/e}(t)$;

\subsection{Kolorowanie krawędzi}

\entry
Funkcja $f: E[G] \rightarrow \set{1, \dots, k}$ to kolorowanie krawędziowe,
  jeśli kraw. incydentne mają różne kolory. Indeks chromatyczny $\chi_e(G)$ to
  najmniejsze $k$, dla którego istnieje $k$-kolorowanie kraw.;

\entry
Tw. Vizinga: $\forall_G \chi_e(G) \leq \Delta(G) +1$;

\entry
Tw. (K{\"o}nig): $G$ dwudzielny, to $\chi_e(G) = \Delta(G)$;

\subsection{Systemy różnych reprezentantów}

\entry
SRR dla rodziny zbiorów $\angles{A_i}_{i\in I},$ to ciąg elem.
  $\angles{a_i}_{i\in I}$ t., że
  $\forall_{i\in I} a_i \in A_i$ oraz $a_i \neq a_j$
  (skojarzenia w g. dwudzielnym);

\entry
Tw. (Hall): SRR dla skończonej r. zb. skończonych $\angles{A_i}_{i=1}^n,$
  istnieje
  $\iff \forall_{J\subseteq\set{1,\dots,n}} |\bigcup_{j\in J}A_j| \geq |J|$;

\entry
$G$ dwudzielny $r$-regularny $\Rightarrow r$-kolorowalny kraw.;

\entry
$G$ dwudzielny, regularny ma pełne skojarzenie;

\entry
$G$ $(n-m)$-regularny $\Rightarrow \exists$ pełne skojarzenie;

\entry
Tw.: Podziały $\mathcal{A}$ i $\mathcal{B}$ mają wspólny SRR $\Leftrightarrow
  \forall_{J\subseteq I} |\bigcup_{j\in J}g(A_j)| \geq |J|$, gdzie
  $g(C) = \set{j | C \cap B_j \neq \emptyset}$;

\entry
$A_1 \cup \dots \cup A_n = B_1 \cup \dots \cup B_n$ i
  $\forall_{1 \leq i \leq n}|A_i| = |B_i| =r \Rightarrow \mathcal{A}$ i
  $\mathcal{B}$ mają SRR;

\input{src/teoria-liczb.tex}
\input{src/teoria-grup.tex}

\end{document}
